\documentclass[oneside,a4paper]{article}
%\usepackage{lectures}
\usepackage{xr}
\usepackage{ulem}
\usepackage{amsmath}
\externaldocument{all-lectures}

\begin{document}
%\includeonly{}
%\intput{all-lectures.tex}
\title{Extra credit problems}
\author{Math 485}
\date{}
\maketitle

\textit{}

\ 

\noindent {0.} Find a mistake or misprint in ``Extra pearls''.
(The score depends on the type of mistake.)


\ 

\noindent %\sout
{1.} %\textit{(solved)
Assume $d_1,\dots, d_p$ is a sequence of integers in a nonincreasing order.
Show that it is multigraphic if and only if $d_p\ge0$, the sum $d_1+\dots+ d_p$ is even and 
\[d_1\le d_2+\dots+d_p.\]
(A sequence of integers $d_1,\dots, d_p$ is called \textit{multigraphic} if it appears as a sequence of degrees of a multigraph.)

\ 

\noindent %\sout
{2.} %\textit{(solved)
Assume that the sequence $d_1,\dots, d_p$ is graphic,
$d_i\ge 1$ for each $i$ and 
\[d_1+\dots+d_p\ge 2\cdot(p-1).\]
Show that there is a connected graph $G$ with the degree sequence $d_1,\dots, d_p$.


\ 

\noindent \sout
{3.} \textit{(solved)} 
Show that in any connected graph $G$ there is a vertex $v$ such that $G-v$ is connected.

\ 


\noindent \sout
{4.} \textit{(solved)} 
Let $G$ be a  connected graph.
Show that any two paths of maximum length in $G$ have a common vertex.

\ 

\noindent %\sout
{5.} %\textit{(solved)} 
Assume two trees $R$ and $S$ 
have the vertices $r_1,\dots,r_n$ and $s_1,\dots,s_n$ correspondingly.
Assume that $R-r_i$ is isomorphic to $S-s_i$ for each $i$.
Show that $R$ is isomorphic to $S$.

\ 

\noindent \sout
{6.} \textit{(solved)}
Let $G$ be a critical graph and $\chi(G)=k+1$.
Show after removing any $k-1$ edges from $G$  
the obtained graph remains connected. 


\ 

\noindent %\sout
{7.} %\textit{(solved)}
Assume both sequences $d_1,\dots, d_p$
 and $d_1-1,\dots, d_p-1$ are graphic.
Show that there is a graph with a 1-factor and with the degree sequence $d_1,\dots, d_p$.

\ 


\noindent %\sout
{8.} %\textit{(solved)} %!!!it is in the book
Show that any $4$-regular graph has a $2$-factor.

\ 


\noindent %\sout
{9.} %\textit{(solved)}
Show that any edge of cubic graph lies in an even number of Hamiltonian cycles.


\ 

\noindent \sout
{10.} \textit{(solved)} 
Assume that the edges of a complete graph are colored in two colors.
Show that there is a Hamiltonian cycle which either monochromatic or consists of two monochromatic paths.

\ 

\noindent %\sout
{11.}  %\textit{(solved)} 
Assume that $G$ is a $3$-regular connected graph such that 
for any two vertices $v,w$ of $G$ there is an isomorphism
$G\to G$ which sends $v$ to $w$.
Prove that $G$ remains connected after removing any $2$ edges.

\ 


\noindent %\sout
{12.} %\textit{(solved)}
Let $G$ be a connected graph.
Given any two vertices $v,w$ in $G$, denote by $d(v,w)$ the length of a shortest path containing $v$ to $w$. 

Assume that $G$ has no triangles and
\[d(x,y)+d(v,w)=\max\{\,d(x,v)+d(y,w),d(x,w)+d(y,v)\,\}\]
for any 4 vertices $x,y,v,w$ in $G$.
Show that $G$ is a tree.

\ 

\noindent %\sout
{13.} %\textit{(solved)}
Understand the proof of Vizing's theorem (see the references for Theorem 2.2.2 in ``Pearls'' or find a proof elsewhere).

\ 

\noindent %\sout
{14.} %\textit{(solved)}
Suppose $p=r(m,n)$ is the Ramsey number for $m$ and $n$%;
%that is $p$ is the minimal integer such that any edge-coloring of the
%complete graph with $K_p$ using red and blue must contain a red $K_m$ or a blue $K_n$
.

Assume that $G$ results from $K_p$ by deleting a single edge.
Show that $G$ has a red/blue edge coloring with no red
$K_m$ and no blue $K_n$.

\ 

\noindent \sout{15.}\textit{(solved)} Show that a graph $G$ with $p$ vertexes is a tree if and only if
its chromatic polynomial is $k\cdot(k-1)^{p-1}$.

\ 

\noindent {16.} Let $T$ be a spanning tree in a graph $G$ with weighted edges.
Show that $T$ has minimal total weight if and only if the weight of any edge $(a,b)$ in $G$ is at least as large as the weight of any edge on the path from $a$ to $b$ in $T$. 

\ 

\noindent {17.} In a group of people, for some fixed $s$ and any $k$,
any $k$ girls like at least $k-s$ boys in total.
Show that then all but $s$ girls may get married on the boys they like.

\end{document}



\ 

\noindent \sout{17.} %\textit{(solved)} 
Let $M$ be a connected planar map and $G$ be its dual.
Show that $M$ and $G$ have the same number of spanning trees.



\ 


\noindent {12.} %\textit{(solved)}
Let $M$ be a maximal planar map.
Assume the vertcies of $M$ are colored in 3 colors.
Show that the number or regions which get all three colors is even.

\ 


\noindent %\sout
{13.} %\textit{(solved)}
Assume a graph $G$ is decomposed into two subgraphs $H_1$ and $H_2$.
Show that 
\[\chi (G)\le \chi(H_1)\cdot \chi(H_2).\]
Given two positive integers $m$ and $n$ construct a graph $G$ with a decomposition into two subgraphs $H_1$ and $H_2$ such that
\[\chi(H_1)=m, 
\quad 
\chi(H_2)=n
\quad\text{and}\quad 
\chi(G)=m\cdot n.\]

\ 